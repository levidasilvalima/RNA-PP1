\begin{abstract}
    This work aims to make a study on the COVID-19 case database in the minicipio de manaus taking into account a period from 01/04/2020 to 07/31/2020. For this, the development tools present for Python were used with the help of Google Colab development environment, which gathers devices for analyzing large amounts of data. The auxiliary tools used were pyplot, plotly, pandas and spicy which were used for specific calculations, data handling and graph plotting.
\end{abstract}
     
\begin{resumo} 
   Este trabalho tem por objetivo fazer um estudo sobre a base de dados de casos de COVID-19 no município de Manaus levando em consideração um período que vai de  01/04/2020 à 31/07/2020. Para tal foram utilizadas as ferramentas de desenvolvimento presentes para Python  com o auxilio do ambiente de desenvolvimento Google Colab que reúne aparatos para analise de grande quantidade de dados. As ferramentas auxiliares utilizadas foram pyplot, plotly, pandas e spicy para cálculos específicos, manipulação de dados e plote de gráficos.
\end{resumo}