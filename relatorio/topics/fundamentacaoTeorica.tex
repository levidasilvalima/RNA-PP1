\textit{Python} é uma linguagem de programação de alto nível,interpretada, de \textit{script}, imperativa, orientada a objetos, funcional, de tipagem dinâmica e forte. Foi lançada por Guido van Rossum em 1991. Atualmente possui um modelo de desenvolvimento comunitário, aberto e gerenciado pela organização sem fins lucrativos Python Software Foundation. Apesar de várias partes da linguagem possuírem padrões e especificações formais, a linguagem como um todo não é formalmente especificada.

Dentre as diversas bibliotecas presentes no \textit{python}, para esse projeto vale ressaltar: \textit{pandas}, \textit{matplotlib}, \textit{plotly}, \textit{math}, \textit{scipy}, \textit{datetime} que são usadas para fazer cálculos específicos e também para construir imagens e referencias interativas em forma de gráficos sobre um conjunto de informações.

Por fim o ambiente de codificação Google Colab que se reuniu como um coletivo em 1977, usando pela primeira vez o nome Green Corporation, e inicialmente recebeu uma bolsa \textit{nea workshop} através do Center for New Art Activities, Inc. Um pequeno sem fins lucrativos formado em 1974 que reúne todas essas ferramentas de forma a permitir o uso sem qualquer restrição por parte dos usuários